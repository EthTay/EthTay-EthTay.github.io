%%%%%%%%%%%%%%%%% PREAMBLE %%%%%%%%%%%%%%%%%%%%%%%%%%%%
%Change the font size of your document - 10pt, 12.1pt, etc.
\documentclass[letterpaper,10pt,oneside]{article}
\usepackage[utf8]{inputenc}
\usepackage{setspace}
\usepackage{hyperref}
\usepackage{graphicx}
\usepackage[left=1in, right=1in, bottom=0.5in, top=0.5in]{geometry}

%%%%%%%%%%%%%%%%% END OF PREAMBLE %%%%%%%%%%%%%%%%%%%%%

\begin{document}

%%%%%%%%%%%%%%%%% NAME OF APPLICANT %%%%%%%%%%%%%%%%%%%

\pagenumbering{gobble}
\noindent  \LARGE{\textbf{Curriculum Vitae - Ethan Taylor}}  \\
\vspace{-2ex}
\hrulefill \\[2mm]
\normalsize
\\[-3mm]
%%%%%%%%%%%%%%%%% CONTACT INFORMATION %%%%%%%%%%%%%%%%%
% Your email address, website, and Skype name are links to send email, open your website and add you on Skype. 


\begin{tabular}{l l}
36 Rees Gardens    & \hspace{1in} \href{mailto:E.D.Taylor@hotmail.com}{E.D.Taylor@hotmail.com} \\
 Top Valley  \\
 Nottingham  \\
 NG5 9JJ & \hspace{1in} Phone: 07480470882 \\
\end{tabular}
\\[5mm]
\noindent \begin{tabular}{@{} l l}



 \Large{Education}    & \textbf{Northumbria University} \\
     & MPhys Physics with Astrophysics: 2016 - 2020 \\
     & \indent Year 1 Average: 70\% , Year 2 Average: 73\%, and Year 3 Average: 63\%\\[2mm]
       \indent 4th Year Dissertation: & \parbox{4.9in}{\small{`\textbf{Numerical investigation into non-linear MHD wave behaviour in a stratified, coronal atmosphere}'}} \\[1.5mm]
     & \indent\parbox{4.9in}{\small My final Masters year project was to investigate the behaviour of how Magnetohydrodynamic (MHD) waves behave in both a uniform and a stratified atmosphere. This was done using a pre-built code package by the name of `Lare2d' which is written in the `Fortran-90' language. The data sets that these simulations produced were then visualised and analysed using IDL (interactive data language). This dissertation was under the supervision of Professor James McLaughlin.}\\
     & \\     
     \indent 3rd Year Dissertation: & \small{`\textbf{A Review of theoretical Dark Matter Models}'} \\
     & \indent\parbox{4.9in}{\small My third year dissertation looked at the problem of the missing matter within the universe. A review of the different models that offer an explanation for this ranging from Modified gravity to new particles, where the project ultimately took a closer look at the Axion. The project then looked at the equations that build this model and attempted to link the physical meaning behind the results to possible observations and what environments they would be found in. This dissertation was under the supervision of Dr Eamon Scullion.}\\
     & \\
     & BEng Electrical and Electronic Engineering: 2015 - 2016\\
     & \indent Year 1 Average of 73\% \\[2mm]
     & \textbf{Bilborough College} (A Levels): 2013 - 2015\\
     & \indent A2: Physics, Maths, Electronics \\
     & \indent As: Computing \\
     & \\

\Large{Work Experience} 

%	& \textbf{Northumbria University Events Rep: 2019 - Present}\\
%	& \indent\parbox{4.9in}{\small Giving tours to groups of prospective students around the university and informing them about points of interest}\\
%	& \indent\parbox{4.9in}{\small  Answering questions prospective students would have about the course}\\
%	& \indent\parbox{4.9in}{\small  Speaking about university life and informing prospective students what its like}\\
%	
%			&\\  

	& \textbf{GCSE Student Mentor at Top Valley academy: 2014 - 2015}\\[0.5mm]
	& \indent\parbox{4.9in}{\small  Supported GCSE students with their studies helping them be prepared for their exams}\\
	& \indent\parbox{4.9in}{\small  Taking a group of students and supporting them with revision sessions based around what topics they were\small   struggling with}\\
	& \indent\parbox{4.9in}{Planned revision sessions based around what struggling students needed help with}\\
  
 	& \\

  \Large{Skills} & \parbox{4.9in}{\small  C, C\#, Python, MATLAB, Fortran-90, VB.NET, \LaTeX, Various Microsoft Office \\Packages }\\
  
		&\\  
  
  \Large{Achievements} & \parbox{4.9in}{3x Progression Scholarship.}\\[1.5mm]
  
  &\parbox{4.8in}{10th out of 531 participants in the UK, 29th out of 4514 globally in the Credit Suisse global coding challenge 2019.}\\
  
  		&\\  
  
  \Large{Sports} & \parbox{4.9in}{\textbf{Northumbria University}} \\
   
    & University Boxing Team Captain: 2019 - Present\\
  	& \indent\parbox{4.9in}{\small Relaying information between the university and the club in a timely manner}\\
	& \indent\parbox{4.9in}{\small Making sure members filled out any paperwork that is needed to compete}\\
	& \indent\parbox{4.9in}{\small Making sure all paperwork relating to attendance and health and safety is filled out properly each session}\\
  
\end{tabular}
\vfill
%%%%%%%%%%%%%%%%% REFERENCES %%%%%%%%%%%%%%%%%%%%%%%%%%
% The reference section has links to your references' websites and email addresses.

\hrulefill \\[2mm]
\noindent \begin{tabular}{@{} l l l}
 \Large{References} & \href{https://www.northumbria.ac.uk/about-us/our-staff/s/eamon-scullion/}{Dr Eamon Scullion} & \href{https://www.northumbria.ac.uk/about-us/our-staff/l/rodrigo-ledesma-aguilar/}{Dr Rodrigo Ledesma-Aguilar} \\
 & \small{Mathematics, Physics and Electrical Engineering} &  \small{Mathematics, Physics and Electrical Engineering}  \\
 & \small{Northumbria University} &  \small{Northumbria University} \\
 & \small{\href{mailto:Eamon.Scullion@Northumbria.ac.uk}{Eamon.Scullion@Northumbria.ac.uk}}& \small{\href{mailto:Rodrigo.Ledesma@Northumbria.ac.uk}{Rodrigo.Ledesma@Northumbria.ac.uk}}
\end{tabular}

\end{document}
